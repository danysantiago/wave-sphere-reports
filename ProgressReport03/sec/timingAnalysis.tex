\section{Timing Analysis}
The timing analysis consists of three main parts: Time bases, Point-to-point communication, and analog considerations. The purpose of this analysis is to ensure that no signal is transmitted at a faster frequency than the one it's receiver can read, thus guaranteeing communications functionality between all components and the MCU.

\subsection{Time bases}
This section compares the maximum external clock frequencies of all components and their communication ports to make sure that no component will be overclocked by the MCU and that all synchronous communication can be performed at the speed of the MCU's clock. Table \ref{tab:compFreq} shows the frequency specifications for the components.  Since the sample frequency will be a relatively low 256Hz, there is complete compatibility between the MCU and all components. The MCU can run at a clock frequency of up to 16MHz. For I$^2$C communications, the standard of 10-100kHz will be used, since all components are compatible with it. For SPI communications, a rate of up to 2MHz can be used, since the slowest device, the RF wakeup receiver, operates at this frequency. The only component that will be communicating with the MCU through UART is the GPS receiver, which will be communicating at a 9600 baud rate and tolerates a 10\% error.

Two timers will be needed for sampling. The first timer will control the sample frequency for all the components.  In order to sample at an accurate frequency of 256Hz, an external crystal with a low frequency of 38.400kHz will be used. To achieve this, the crystal will be connected to one of the timers, and the terminal count will be set to 150 with a prescale divider of one. The second timer will control the sampling time of 30 seconds. The prescaler of this timer will be set to 1 and the terminal count will be set to 38,400. 30 counts of the terminal register will produce the interrupt that will conclude the sampling mode.

The reason why a 38.400kHz crystal was chosen is because the GPS communicates through UART with a 9600 baud rate. If a standard 32.768kHz crystal is used instead, the frequency divider would be 3.4, which would produce an effective baud rate of 9637.65bps. This translates to a maximum error of 17.19\%, which is unacceptable to the GPS. This problem is solved by using a 38.400kHz crystal, which would produce an effective baud rate of 9600bps with 0.00\% error with a frequency divider of 4. The use of a 38.400kHz crystal will not affect real time keeping nor sampling rate, because a 256Hz signal can be obtained by counting 150 cycles of the crystal, and a second can be obtained by counting 38400, both of which fit into a 16bit timer compare register.

\begin{table}[H]
  \centering
  \caption{Components Frequencies}
    \begin{tabular}{|c|c|c|}
    \hline
    Component & Protocol & Frequency \\
    \hline \hline
    MCU   & -     & 4, 8, or 16 MHz \\ \hline
    Crystal & -     & 38.400kHz \\ \hline
    ADC   & -     & 200ksps \\ \hline
    Accelerometer & Analog & 500Hz \\ \hline
    Battery Gauge & I$^2$C & 10-100kHz \\ \hline
    Magnetometer & I$^2$C & 0-100kHz \\ \hline
    Gyroscope & SPI   & 0-10MHz \\ \hline
    RF Wakeup & SPI   & 0-2MHz \\ \hline
    SD card & SPI   & 0-25MHz \\ \hline
    Xbee  & SPI   & 0-5MHz \\ \hline
    GPS   & UART  & 9,600 baud \\ \hline
    USB-to-SD &       & 26, 19.2, 13, or 12 MHz \\ \hline
    \end{tabular}%
  \label{tab:compFreq}%
\end{table}%

\subsection{Point-to-point communication}
This section outlines the minimum requirements for timing signals of the devices connected through GPIO. These timing signals do not include forms of serial communication, including UART, SPI and I$^2$C.


\subsection{Analog considerations}
There is only one analog component in the system, the ADXL377 accelerometer. Care must be taken to ensure that the sampling rate of the MCU's ADC can be set to twice the frequency of the analog signal. The ADXL377 can output data at 1000Hz, but this system requires it to be 500Hz. This means that the MCU's ADC must be capable of sampling at a rate of 1ksps. The MSP430FR5969's ADC is capable of sampling at 200ksps, which means that the accelerometer is also compatible with the MCU.

A second consideration should be made when interfacing with an analog component: slew rate compatibility. There is no documented specification of the ADC's slew rate in the MSP430FR5969's datasheet, but the voltage range is specified to be from 0V to +Vcc. As long as the output values of the ADXL377 accelerometer are within the ADC's input range, slew rate will be unaccounted for.


