\section{Timing Analysis}

Table \ref{tab:compFreq} shows the frequency specifications for the components.  Since the sample frequency will be 256Hz, there is complete compatibility between all components.  Two timers are needed.  The first timer will control the sample frequency for all the components.  In order to have 256Hz, an external crystal with frequency of 32,768Hz will be used.  The terminal count will be set to 128 and the prescaler to one.  The second timer will control the sampling time (30s).

%    \begin{tabular}{|m{1in}|m{2.2in}|m{2.2in}|}
\begin{table}[H]
  \centering
  \caption{Component Frequencies}
    \begin{tabular}{|c|c|}
    \hline
    Component & Frequency \\
    \hline \hline
    MCU   & 1, 2.667, 3.5, 4, 5.333, 7, 8, 16 MHz \\ \hline
    Crystal & 32,768 Hz \\ \hline
    Accelerometer & Analog, ADC 200ksps, 1000Hz max bandwidth \\ \hline
    Battery Gauge & I$^2$C, 10-100kHz and 400kHz \\ \hline
    GPS   & UART, 9,600 baud (std), from 4,800 to 921,600 baud. 9,600 baud selected\\ \hline
    Gyroscope & SPI, up to 10MHz \\ \hline
    Magnetometer & I$^2$C, up to 100kHz std, 400kHz fast \\ \hline
    RF Wakeup & SDI (SPI) - 2MHz \\ \hline
    SD card & 0-25MHz std, up to 50Mhz max. \\ \hline
    Xbee  & SPI, no specification \\ \hline
    \end{tabular}
  \label{tab:compFreq}
\end{table}

In order to make sure that the Real Time Clock (RTC) will be precise, the external crystal, with frequency of 32,768 will be used.

%TODO determine error of baud rate in GPS

%TODO Point-2-Point Analysis....


