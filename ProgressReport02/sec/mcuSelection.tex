\section{MCU Selection}
One of the biggest constraint in the system is the size requirement.  Because the spheres are so small, the design forces the use of a small battery. The small battery constraint means that the MCU selected for the application must be a low power model in order to prolong battery life. The space constraint also dictates that the use of fewer integrated circuits (ICs)is preferred, which is why an MCU having integrated peripherals such as an ADC and Real Time Clock (RTC) is required in order to save board space. 

A search has been performed on the low power families of various manufacturers. Microchip's low power MCUs have been considered, specifically the PIC24FV32K family. It contains all the required features and has a low pin count, which helps minimize the space used on the printed circuit board (PCB), but has a significantly higher power consumption than Texas Instrument's (TI) ultra low power MCU line, the MSP430. The MSP430F5359 MCU has everything required plus some very convenient features, like a battery backed up RTC, 12-bit ADC, large amount of memory and extra I/O ports, but compared with TI's other MCUs, this translates to a higher price, higher power consumption, and a very high pin count (100 pins) for this particular application. 

This application requires the system to have a wireless communication interface for system management, data transferring and system wake up from deep sleep, so an MCU with integrated RF capabilities was considered such as TI's CC430F5147. However, the MCU does not have the necessary ports to interface with the selected peripheral components, specifically it is missing an additional UART port required for the selected GPS module. The CC430 was one of the top choices because of the RF receiver integration, so much so that even other GPS modules with different interfaces were being considered. However, all the options found consumed much more power and were big compared to the selected GPS unit.  Furthermore, the CC430 has a general multi-purpose wireless solution which consumed too much power when operating in listening mode, thus the CC430 MCU was discarded when a more specific IC, the AS3930, was discovered that could wake-up the system while still consuming very little power when listening for a wake-up signal. 

Finally, the MSP430FR5969 was considered. This MCU contains FRAM non-volatile memory, combining the speed and endurance of SRAM with the stability of flash. It includes everything that the application requires, like a 12-bit ADC in a 48 pin package, and can be combined with the low power RF wake-up receiver to have the desired functionality of the CC430 while taking advantage of the low power consumption of the MSP430. Thus, the MSP430FR5969 wasfound to be the most suitable option among those considered and will be used to implement the spheres. Table~\ref{tab:mcuComp} shows some the MCUs considered along with some of their technical specifications.
% Table generated by Excel2LaTeX from sheet 'Sheet1'
\begin{table}[H]
\setlength{\extrarowheight}{1.5pt}
  \centering
  \caption{MCU Comparison Table}
    \begin{tabular}{|m{1.5in}|c|c|c|c|}
    \cline{2-5}
    \multicolumn{1}{c|}{} & \begin{sideways}MSP430FR5969 \end{sideways} &\begin{sideways} CC430F5147 \end{sideways}& \begin{sideways}MSP430F5359 \end{sideways}& \begin{sideways}PIC24F32KA302\end{sideways}\\ 
    \hline
    Frequency (MHz) & 16    & 20    & 20    & 32\\ \hline
    FRAM (KB) & 64    & -     & -     & -\\ \hline
    Flash (KB) & -     & 32    & 512   & 32\\ \hline
    SRAM (KB) & 2048  & 4096  & 67584 & 2048 \\ \hline
    GPIO  & 40    & 30    & 74    & 24    \\ \hline
    Timers 16-bit & 5     & 2     & 4     & 5     \\ \hline
    Real-Time Clock & YES   & YES   & YES, battery backup & YES   \\ \hline
    USCI\_A (UART/LIN/IrDA/SPI) & 2     & 1     & 3     & 2     \\ \hline
    USCI\_B (I$^2$C/SPI) & 1     & 1     & 3     & 2     \\ \hline
    ADC   & 12-bit SAR & 10-bit & 12-bit SAR & 12-bit \\ \hline
    ADC Channels & 16    & 6     & 16    & 13    \\ \hline
    Package & 48VQFN & 48VQFN & 100LQFP & 28VQFN \\ \hline
    Pin Count & 48    & 48    & 100   & 28    \\ \hline
    Current Consumption ($\mu$A at 8MHz) & 800   & 1280  & 2360  & 2000  \\ \hline
    Unit Price & \$4.45  & \$3.15  & \$10.40 & \$3.41  \\ \hline
    \end{tabular}%
  \label{tab:mcuComp}%
\end{table}%
