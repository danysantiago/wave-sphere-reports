\section{MCU Selection}
The space constraints of the system that is being designed forces the use a small battery. The small battery constraint means that the micro controller unit (MCU) for the application must be a low power model in order to save battery life. The space constraints means that integrated circuits must be used as much as possible and thus an MCU with internal peripherals such as an ADC and RTC are required to save board space. A search has been performed on the low power families of various manufacturers. Microchip’s low power MCUs have been considered, specifically the PIC24FV32K family. It contains all the features we need and has a low pin count, which helps avoid the space constraint, but it has a significantly higher power consumption than Texas Instrument's ultra low power MCU line, the MSP430. The MSP430F5359 MCU has everything required plus some very convenient extra features, like a battery backed up RTC, 12bit ADC, large amount of memory and extra IO ports, but this translates to a higher price, higher power consumption, and 100 pin count. The application requires for the system to have a wireless communication interface for system management, data transferring and waking up from deep sleep, so an MCU with integrated RF capabilities was considered such as TI’s CC420F5147. However the MCU had a general multi-purpose wireless solution which consumed too much energy when operating in listening mode, thus the MCU was discarded when a more specific IC was discovered that could wake-up the system and would still consume very low power when listening for a wake-up signal. Another considered MCU was the MSP430FR5969. This MCU contains FRAM nonvolatile memory, combining the speed and endurance of SRAM with the stability of flash. It includes everything that the application requires, like a 12bit ADC in a 48 pin package. It is superior to the MSP430FR5721, which comes in a smaller 40 pin package and consumes less memory, but contains a 16th amount of the FRAM and a 10bit ADC as opposed to a 12bit one. Thus, it has been decided that the MSP430FR5969 MCU will be used. On Table~\ref{tab:mcuComp} are the MCUs considered along with some information.
% Table generated by Excel2LaTeX from sheet 'Sheet1'
\begin{table}[htbp]
  \centering
  \caption{MCU Comparation Table}
    \begin{tabular}{|c|c|c|c|c|}
    \hline
    Microcontroller & MSP430FR5969 & CC430F5147 & MSP430F5359 & PIC24F32KA302\\ \hline
    \hline
    Frequency (MHz) & 16    & 20    & 20    & 32\\ \hline
    FRAM (KB) & 64    & -     & -     & -\\ \hline
    Flash (KB) & -     & 32    & 512   & 32\\ \hline
    SRAM (KB) & 2048  & 4096  & 67584 & 2048 \\ \hline
    GPIO  & 40    & 30    & 74    & 24    \\ \hline
    Timers 16-bit & 5     & 2     & 4     & 5     \\ \hline
    Real-Time Clock & YES   & YES   & YES, battery backup & YES   \\ \hline
    USCI\_A (UART/LIN/IrDA/SPI) & 2     & 1     & 3     & 2     \\ \hline
    USCI\_B (I$^2$C/SPI) & 1     & 1     & 3     & 2     \\ \hline
    ADC   & 12-bit SAR & 10-bit & 12-bit SAR & 12-bit \\ \hline
    ADC Channels & 16    & 6     & 16    & 13    \\ \hline
    Package & 48VQFN & 48VQFN & 100LQFP & 28 pin QFN \\ \hline
    Pin Count & 48    & 48    & 100   & 28    \\ \hline
    Current Consumption ($\mu$A at 8MHz) & 800   & 1280  & 2360  & 2000  \\ \hline
    Unit Price & \$4.45  & 3.15  & \$10.40 & \$3.41  \\ \hline
    \end{tabular}%
  \label{tab:mcuComp}%
\end{table}%
