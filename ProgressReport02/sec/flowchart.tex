\section{Operating Charts}
Operating charts describe the modes of operation and the flow of the system without much detail. On Figure~\ref{fig:systemFlowchart} is the system's main flowchart, in it one can see how the device goes in and out of each of its operating modes. When the system is powered up it performs its Boot Sequence, which includes initializing the MCU and all of its peripherals. Once the Boot Sequence has been completed the device will enter Shutdown Mode. In Shutdown Mode the system is operating with very low power and is also listening for a specific RF wake-up signal to further interact with the user. When the device receives the correct wake-up signal it will enter Stand-By Mode. On Stand-By Mode the device establishes a ZigBee connection with the base station. Once connected the user can then select the next operating mode. The available selectable operating modes are Diagnostic, Sampling and Data Retrieval modes. The system is not intended for a full power down, but it can occur if the battery is fully depleted, in this case when the device is powered back from such power down it will initialize it's boot sequence. Further explanation of the operating modes can be found in each of the modes section.
\begin{figure}[H]
	\centering
	\includegraphics[scale=0.8]{img/SystemFlowchart}
	\caption{System Flowchart \label{fig:systemFlowchart}}
\end{figure}

\subsection{Boot Sequence}
The boot sequence occurs when the device is powered on from a cold power down. The boot sequence contains the functions and routines to initialize the device and its peripherals. The system power ups and performs a quick test on each component, if a successful boot sequence occurs the device heads into Shutdown Mode. However, if any of the components fail to respond and the device is not suited for further use, a Red LED indicator will be activated to notify the user.
\begin{figure}[H]
	\centering
	\includegraphics[scale=1.0]{img/BootSequenceFlowchart}
	\caption{Boot Sequence \label{fig:bootSequence}}
\end{figure}

\subsection{Shutdown Mode}
On shutdown mode the device is in deep sleep, the MCU is in shutdown mode and all components except one are turned off. The single component that will be powered on is the RF Wake-up receiver. The low power ASK receiver will be operating in listening mode. On this mode the receiver is capable of detecting the wake-up carrier while consuming very little power. When the carrier is detected the receiver sends an interrupt signal to the MCU, waking it up and proceeding to enter Stand-By Mode.
\begin{figure}[H]
	\centering
	\includegraphics[scale=1.0]{img/ShutdownMode}
	\caption{Shutdown Mode \label{fig:shutdownMode}}
\end{figure}

\subsection{Stand-By Mode}
On Stand-By mode the system is conserving power with some components turned off while maintaining a ZigBee connection to the base station. On this mode the system is waiting for user interaction, the device is connected to the base station and simple information is transmitted to the base station's user interface. The transmitted information is battery level and SD card space usage. The system is also waiting for the user to select the next operating mode via the graphical interface on the base station. The use can either test the device with Diagnostic Mode, retrieve its data with Retrieval Mode, or enter Sampling Mode and run an experiment. The components that are awake on this mode are the SD card module and the XBee module.
\begin{figure}[H]
	\centering
	\includegraphics[scale=1.0]{img/StandByMode}
	\caption{Stand-By Mode \label{fig:standByMode}}
\end{figure}

\subsection{Diagnostic Mode}
On Diagnostic Mode the device maintains communication to the base station and sends real-time information of the different components to the base station, the user can then observe and verify such data.
\begin{figure}[H]
	\centering
	\includegraphics[scale=1.0]{img/DiagnosticMode}
	\caption{Diagnostic Mode \label{fig:diagnosticMode}}
\end{figure}

\subsection{Sampling Mode}
On Sampling Mode the device captures data from the accelerometer, gyroscope and magnetometer. The time the system is capturing data is defined by the user through the base station. On this mode all other components are turned off except those from which data will be captured and the SD card which is where the data will be saved. Data is captured for the duration of the previously specified time and data points are saved in the SD card, each with a time stamp from the time which they were captured. Once the data capturing time is over, the device automatically starts operating in Locate Mode.
\begin{figure}[H]
	\centering
	\includegraphics[scale=1.0]{img/SamplingMode}
	\caption{Sampling Mode \label{fig:samplingMode}}
\end{figure}

\subsection{Locate Mode}
On Locate Mode the device is trying to be found by the user. The system turns on the XBee Module and the GPS and turns off all other components. The system then proceeds to establishing a ZigBee connection to the base station and get a GPS lock. Once the device gets its location using the GPS Module it will then broadcast the location to the base station trough the ZigBee connection. When the user successfully retrieves the device it can be switched to Stand-By Mode for later use.
\begin{figure}[H]
	\centering
	\includegraphics[scale=1.0]{img/LocateMode}
	\caption{Locate Mode \label{fig:locateMode}}
\end{figure}

\subsection{Retrieval Mode}
Retrieval Mode is the operation mode of the device in which the system transfers the collected data from Sampling Mode saved in the SD Card to the base station. Data is transferred trough the established ZigBee connection and once all data has been transferred it is erased from the SD card.
\begin{figure}[H]
	\centering
	\includegraphics[scale=1.0]{img/RetrievalMode}
	\caption{Retrieval Mode \label{fig:retrivalMode}}
\end{figure}



