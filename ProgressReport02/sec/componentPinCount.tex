\section{Preliminary Components and Pin Count}

Table \ref{tab:sensorResolution} shows a list of specifications and minimum resolution required for each sensor.

% Table generated by Excel2LaTeX from sheet 'Sheet1'
\begin{table}[H]
\setlength{\extrarowheight}{1.5pt}
  \centering
  \caption{Component Specifications}
    \begin{tabular}{|m{1in}|m{2.2in}|m{2.2in}|}
    \hline
     Part  &  Specifications &  Resolution \\
    \hline \hline
    Accelerometer & Must be at least +/- 200g & Accuracy of at least 0.1g \\ \hline
    Gyroscope & At least 2000dps & Accuracy of at least 0.1dps \\ \hline
    Magnetometer & & Accuracy of at least 1$^{\circ}$ \\ \hline
    GPS   &  & Accuracy to within 3m  \\ \hline 
    XBee  & & \\ \hline
    
    \end{tabular}%
  \label{tab:sensorResolution}%
\end{table}%

Table \ref{tab:componentPinCount} shows a list of the essential components in the system along with their model number, supply voltage level, interface to the microcontroller and pin count.  This table is not meant to be exhaustive, it is simply a preliminary list to facilitate counting the pins required on the microcontroller. The components were chosen to satisfy the requirements specified in Table \ref{tab:sensorResolution}.

%TODO Add missing components
% Table generated by Excel2LaTeX from sheet 'Sheet1'
\begin{table}[H]
\setlength{\extrarowheight}{1.5pt}
  \centering
  \caption{List of essential components}
    \begin{tabular}{|c|c|m{0.35in}|c|c|}
    \hline
    Part  & Model & \centering Vcc & Interface & Pins Required \\
    \hline \hline
    Accelerometer & ADXL377 & \centering 3.3V  & Analog & 3 analog \\ \hline
    Gyroscope & L3GD20 & \centering 3.3V  & SPI   & 3 SPI, 1 GPIO (Chip Select) \\ \hline
    Magnetometer & 511-LSM303DLM & \centering 3.3V  & I$^2$C   & 2 I$^2$C, 1 GPIO \\ \hline
    GPS   & IT520 & \centering 3.3V   & UART  & 2 UART, 3 GPIO \\ \hline 
    XBee  & XBee-PRO ZB SMT &\centering 3.3V  & SPI   & 3 SPI, 3 GPIO \\ \hline
    SD Module &   \centering -    & \centering 3.3V  & SPI   & 3 SPI, 1 GPIO (Chip Select) \\ \hline
    RF Wakeup & AS3930 & \centering 1.8V  & SPI   & 2 GPIO, 3 SPI \\ \hline
    Light Sensor & \centering - & \centering - & Analog & 1 analog \\ \hline
    Battery Meter & BQ27200 & 0-7V & \centering I$^2$C & 2 I$^2$C \\ \hline
    3 LEDs & \centering - & \centering - & GPIO & 3 GPIO \\ \hline
    
    \end{tabular}%
  \label{tab:componentPinCount}%
\end{table}%

Since the SPI standard allows for the connection of multiple components with the same three clock, data-in and data-out lines, all the devices that have an SPI interface will be connected in parallel.  This allows for the use of less microcontroller pins while creating a simple and elegant design.  An individual pin has been assigned for the chip select lines needed for each component.  Although less pins could be used for the chip-select lines, the aforementioned approach was used in order to save space on the board by forgoing the need of using another integrated chip for a demultiplexer.  After taking this into account, the pin count is as follows:

%TODO Count Pins
% Table generated by Excel2LaTeX from sheet 'Sheet1'
\begin{table}[H]
\setlength{\extrarowheight}{1.5pt}
  \centering
  \caption{Total Pin Count}
    \begin{tabular}{|c|c|}
    \hline
    Pin Type  & Count \\
    \hline \hline
   	SPI & 3 \\ \hline
   	I$^2$C & 2 \\ \hline
   	UART & 2 \\ \hline
   	GPIO & 14 \\ \hline
   	ADC Channels & 4 \\ \hline \hline
   	Total & \\ \hline
    \end{tabular}%
  \label{tab:pinCountTotal}%
\end{table}%

