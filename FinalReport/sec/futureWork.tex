\section{Future Work}
\label{sec:futureWork}
Although a complete functional prototype was delivered, thorough field testing has yet to be performed because of time constraints.   The calibration of the sensors can be different from board to board because of the different amounts of stress caused by the soldering process.  Since data conversion takes place in the Graphical User Interface this means that a calibration mode should be added to the software.  The software should take at least two parameters for each sensor axis: a scaling factor and an offset.  Although there are very elaborated calibration processes and methods, these two parameters are essential to any calibration process. 

From a software perspective, more measures to protect data integrity should be implemented since for this type of application having corrupted data is unacceptable.  Some measures might include storing data redundantly, oversampling and averaging values on other sensors, among others.

In addition to this, during the late stages of the prototype development one of the researchers discovered from his gathered data that the gyroscope was very sensible to vibrations and changes in acceleration, something that will always occur because of the nature of the experiments being conducted with these drifters.  The gyroscope used for this prototype suffers from the same negative effect as the one the researcher is using.  Because of the advanced stage of development, it became unfeasible to change the selected gyroscope to one that was more resistant to these effects. Therefore, in a future version of the prototype a new gyroscope that is shock resistant needs to be chosen in order to yield better results.