\section{Timing Analysis}
The timing analysis consists of three main parts: Time bases, Point-to-point communication, and analog considerations. The purpose of this analysis is to ensure that no signal is transmitted at a faster frequency than the one its receiver can read, thus guaranteeing communications functionality and interoperability between all components and the MCU.

\subsection{Time bases}
This section compares the maximum external clock frequencies of all components and their communication ports to make sure that no component is overclocked by the MCU and that all synchronous communication can be performed at the speed of the MCU's clock. Table \ref{tab:compFreq} shows the frequency specifications for the components.  Since the sample frequency will be a relatively low 250Hz, there is complete compatibility between the MCU and all components. The MCU can run at a clock frequency of up to 16MHz. For I$^2$C communications, the fast mode standard of 100-400kHz is being used, since all components are compatible with it. For SPI communications, the signaling rate can be changed between devices. For the slowest device, the RF wakeup, a rate of 1MHz is being used. For the SD card, a rate of 4MHz was implemented. Two components are communicating with the MCU through UART, the XBee wireless receiver/transmitter and the GPS receiver. Both of them communicate with a baud rate of 9600. Since they must be used at the same time, two separate UART buses were used, and since the MCU did not have enough UART state machines, Software UART through GPIO ports was implemented for the GPS receiver.

\paragraph{Timers}

Three timers are needed in total, two for sampling and one for implementing UART via software. The first sampling timer controls the sample frequency for all the components.  In order to sample at an accurate frequency of 250Hz, an external crystal with a frequency of 12MHz is being used. To achieve this, the crystal feeds the master clock and sub-main clock (MCLK and SMCLK, respectively), and the terminal count is set to 48,000 with a prescale divider of one. The second sampling timer will be used to count the time and update the timestamp whenever a sample is taken. The prescaler of this timer will be set to one and the terminal count will be set to 12,000. Thus, this timer will be ticking at 1kHz, and each count of the timer will correspond to 0.1ms of elapsed time. This is more than enough precision for the intended application. An additional timer is needed for the software UART implementation, which uses the capture/latch feature of the timer to detect incoming data packets and interrupt the MCU. It also uses the interrupt feature for the transmitting function. After a bit is transmitted through the UART bus, the timer is set up to fire at the time in the future when the next bit must be output on the bus.

\paragraph{12MHz Crystal Oscillator}
The reason why a 12MHz crystal was chosen is because the USB-to-SD card reader needs a 12MHz clock input from the MCU for establishing USB communications. Additionally, the GPS and Xbee modules communicate through UART with a 9600 baud rate. If a standard 32.768kHz crystal is used instead, the frequency divider would be 3.4, which would produce an effective baud rate of 9637.65bps. This translates to a maximum error of 17.19\%, which is unacceptable for maintaining a stable communication link with the GPS module. This problem is solved by using a 12MHz crystal, which would produce an effective baud rate of 9600bps with 0.00\% error with a frequency divider of 1250. The use of a 12MHz crystal will not affect real time keeping nor sampling rate, because a 250Hz signal can be obtained by counting 48,000 cycles of the crystal, and a tenth of a millisecond can be obtained by counting 12,000, both of which fit into a 16bit timer compare register. 

\begin{center}
    \begin{longtable}{|c|c|c|}
    \caption{Component Operating Frequencies  \label{tab:compFreq}} \\
     \hline
    \rowcolor{Gray}
   Component & Protocol & Frequency \\
    \hline \hline \endfirsthead
    
         \hline
    \rowcolor{Gray}
    Component & Protocol & Frequency\\
    \hline \hline \endhead
    
    \endfoot
    MCU   & -     & 4, 8, 12, or 16 MHz \\ \hline
    %External Crystal & -     & 12.000MHz \\ \hline
    ADC   & -     & 200ksps \\ \hline %no se si esto hay q ponerlo...
    Accelerometer & Analog & 500Hz \\ \hline
    Battery Gauge & I$^2$C & 10-100kHz \\ \hline
    Magnetometer & I$^2$C & 0-400kHz \\ \hline
    Gyroscope & I$^2$C   & 0-400kHz \\ \hline
    RF Wakeup & SPI   & 0-2MHz \\ \hline
    SD card & SPI   & 0-25MHz \\ \hline
    GPS   & UART  & 9,600 baud \\ \hline
    Xbee  & UART   & 9,600 baud \\ \hline
    \end{longtable}%

\end{center}%
  \vspace{-2cm}
  
  
  
\subsection{Point-to-point communication}
This section outlines the minimum requirements for timing signals of the devices connected through GPIO. Table~\ref{tab:timeReqs} shows the timing requirements for the components. These timing signals do not include forms of serial communication, such as UART, SPI and I$^2$C. If a device requires a specific setup or hold time, the solution would be to set up a timer and count the specific number of clock ticks that the device requires. 

\begin{center}
    \begin{longtable}{|c|p{3in}|}
    \caption{Signal Timing Requirements of Components  \label{tab:timeReqs}} \\

     \hline
    \rowcolor{Gray}
   Signal  & Time Required \\
    \hline \endfirsthead
    
         \hline
    \rowcolor{Gray}
    Signal  & Time Required \\
    \hline \endhead
    
    \endfoot

    GPS\_DR (DR\_INT) & Needs to be toggled by low-high-low with $>$10ms pulse length \\ \hline
    GPS\_FIX (UI\_FIX) & Signal outputs 1s pulse every 2s during valid fix condition \\ \hline
    %GYR\_CS (CS) & Setup time: 5ns, Hold time: 8ns \\ \hline
    GYR\_DR (DRDY/INT2) & Interrupt: enabled until acknowledged by the MCU \\ \hline
    MAG\_DR (DRDY) & Interrupt: enabled when a new set of measurements are available \\ \hline
    RF\_WK (WAKE) & Not specified \\ \hline
    RFWK\_CS (CS) & Needs to be high as long as data needs to be read. 65 clock cycles to calibrate \\ \hline
    SD\_CD (HCRD\_PRST) & Active when card present \\ \hline
    SD\_CS (CS) & Asserted 74 clock cycles \\ \hline
    SU\_BERR (!BERR/INT) & Low when error occurs, stays until error is cleared \\ \hline
    SU\_BUSY (!BUSY) & $>$100ms to complete enumeration/de-enumeration \\ \hline
    SU\_MODE (MODE) & Active during simple control \\ \hline
    %XB\_CS (SPI\_!SSEL) & Not specified \\ \hline %signal not needed anymore
    XB\_DR (!DTR/SLEEP) & Not specified \\ \hline
    XB\_SLEEP (ON/!SLEEP) & Not specified \\ \hline
    \end{longtable}%

\end{center}%
  \vspace{-2cm}
  

\subsection{Analog considerations}
There is only one analog component in the system, the ADXL377 accelerometer. Care must be taken to ensure that the sampling rate of the MCU's ADC can be set to twice the frequency of the analog signal and that the input voltage range of the ADC is enough to allow for the output voltage swing of the ADXL377. The ADXL377 can output data at a maximum of 1000Hz.  This means that the MCU's ADC must be capable of sampling at a rate of at least 2ksps. The MSP430FR5969's ADC is capable of sampling at 200ksps, which is well above the required value.  The acceptable input voltage range is specified to be from 0V to +Vcc. Since the ADXL377 output voltage swing ranges from 0.1V to 2.8V, the accelerometer's output can be sampled without suffering from clipping distortion.  This makes the ADXL377 compatible with the MCU.

A second consideration should be made when interfacing with an analog component: slew rate compatibility. There is no documented specification of the ADC's slew rate in the MSP430FR5969's data sheet. However, since the output of the accelerometer varies around 6.5 mV/g, the change in amplitude in the input signal should be small enough so that there will not be any slew rate distortion.

