\section{Theoretical Background}
\todo[inline]{In Progres...}
In fluid mechanics, there are two ways to describe flow: Eulerian and Lagrangian.  The Eulerian approach describes what is happening at a given location for a given time, whereas the Lagrangian approach describes the history of the particle exactly \cite{Granger1985}.  Looking at these two terms from a sensors point of view, an Eulerian sensor would be at a fixed position recording data over time, whereas a Lagrangian sensor would move with the particles of the fluid while recording data.  It is easy to see that for the study of wave breaking phenomena a Lagrangian sensor would be preferable, as the path of the waves are unpredictable and an initial location cannot be determined before the event occurs.

The drifters are to be released within breaking waves to measure important variables in this process.
%The objective of taking measurements from the IMU is to re-create the sensor's trajectory via dead reckoning.  Mention Ineratial Navigation systems and dead reckoning.

It was determined in \cite{Canals2012} that the sphere diameter to wave height ratio should be minimize, which is why a spherical 7.5 cm diameter capsule was assembled.