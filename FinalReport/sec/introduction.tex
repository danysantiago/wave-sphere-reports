\section{Introduction}

Although waves are abundant at sea and everyone who has been near a shoreline has seen and interacted with a wave, most people don't think about the wave-breaking phenomena when enjoying their time in the water.  However, this topic is very important for the researchers at the Fluid Mechanics and Ocean Engineering Laboratories at the University of Puerto Rico at Mayag\"uez, which are directed by Dr.~Miguel Canals.  The natural physics and motion dynamics of the wave-breaking phenomena have not been thoroughly studied because of the difficulty encountered when trying to measure the characteristics of the waves.

A novel way to study these dynamics is by developing an instrument, also referred to from hereon as drifter, that will ride with the waves and take measurements during the wave-breaking process.  This is the approach that the researches have taken as part of the NSF Funded project titled ``Lagrangian Observations of Turbulence in Breaking Waves".  Currently, the researchers\footnote{From hereon, the word researchers and users will be used interchangeably.} working on this project have an assembled prototype with which they have been performing experiments. This drifter consists of a small spherical plastic casing which houses a series of sensors, a microSD card, a microcontroller and power management components.

However, their prototype has a series of issues that need to be addressed so that the overall functionality is improved.  One example of a current issue is that design being used has an external button to activate the sphere.  The way in which the button was mounted on the plastic spherical casing allowed for the water to leak inside the casing, damaging the electronic components.  Another issue with the current design is the lack of location information. Once the drifter has been deployed and the experiment has concluded, the researchers must retrieve the drifter.  However, they have lost drifters because the waves have taken them away, making them impossible to find.

The objective of this project is to improve the overall design of the drifter currently being used by the researchers by addressing the previously mentioned issues, as well as other issues not mentioned.  Because the drifters will have more capabilities, a base station must also be designed so as to allow for easy management of the individual drifters through a custom application displayed to the user via a graphical interface. 

This work presents the design process of a spherical drifter that will aid the researchers in taking the aforementioned measurements. A system overview is presented so as to give the reader a better idea of the solution proposed by this work.  This is followed by a thorough discussion of the hardware and software design process.  Figures, tables and calculations are presented along the way to support the claims made by this work and aid the reader in understanding certain aspects of the design process.  Finally, a brief recommendation of work that could be done in the future to improve the current prototype is given.  
