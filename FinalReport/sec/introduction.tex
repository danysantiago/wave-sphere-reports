\section{Introduction}

Although waves are abundant at sea and everyone who has been near a shoreline has seen and interacted with a wave, most people don't think about the wave-breaking phenomena when enjoying their time at the beach.  However, this topic is very important for the researchers at the Fluid Mechanics and Ocean Engineering Laboratories at the University of Puerto Rico at Mayag\"uez, which are directed by Dr.~Miguel Canals.  The natural physics and motion dynamics of the wave-breaking phenomena have not been thoroughly studied because of the difficulty encountered when trying to measure the characteristics of the waves.

A novel way to study these dynamics is by developing an instrument, also referred to from hereon as drifter, that will ride with the waves and take measurements during the wave-breaking process.  This is the approach that the researches have taken as part of the NSF Funded project titled ``Lagrangian Observations of Turbulence in Breaking Waves".

This work presents the design of a spherical drifter that will aid the researchers in taking the aforementioned measurements. A thorough explanation of the hardware and software design is presented along with supporting calculations where required.

