\section{Thermal Analysis}
Performing a thermal analysis on the system will reveal whether the operating temperature of the individual ICs is below their maximum rating.  In order to perform this analysis, the junction temperature $T_J$ will be calculated for the MCU and the power ICs.  The junction temperature is given by \cite[419]{Jimenez2013}:

\[T_J = T_A + \theta_{JA} \cdot P_{diss}\]

where $T_J$ is the estimated junction temperature, $T_A$ is the ambient temperature (taken here to be $27^\circ C$), $\theta_{JA}$ is the junction-to-ambient thermal resistance and $P_{diss}$ is the estimated power consumption of the device. 

This formula is applied throughout the following sections to determine whether the MCU and the power ICs will be operating at a safe temperature or if they require additional heat dissipation mechanisms such as heat syncs, fans, etc.

%TODO
\subsection{Microcontroller}
In order to estimate the average operating junction temperature for the MCU the power dissipated by the IC will first be calculated.  The following formula, which was taken from \cite[419]{Jimenez2013}, will be used:

\[P_{diss} = V_{DD} \cdot \left(I_{DD(avg)} + \sum_{allpins} |I_{IO(avg)}| \right)\]

Although no information is available for the $\theta_{JA}$ of the MSP430FR5969, since this value depends on the package and the area exposed to the air, a device with the same package (48-pin QFN), the MSP430F5510, was found to have $\theta_{JA} = 28.6^\circ C/W$ and this value was instead used.  In the same manner, although no information on the IO currents is available, the ones for the MSP430F5510 will be used instead.

In full drive mode, the MSP430F5510 can output upto 45mA through its IO pins while still maintaining a valid $V_{OL}$.    Table \ref{tab:ioCurrents} shows the assumed IO currents for each components.
\begin{table}[H]
  \centering
  \caption{Worst Case IO current for pins}
    \begin{tabular}{|c|c|c|c|}
     \hline
     \rowcolor{Gray}
    Pin & $I_{DD(Active)}$ ($\mu$A) & Qty & Total Current ($\mu$A)\\
     \hline \hline
    SPI & 100  & 3 & 300 \\ \hline
   	$I^2C$ & 10 & 2 & 20  \\ \hline
    UART & 1000 & 2 & 2000 \\ \hline
    GPIO  & 1000 & 21 & 21000\\ \hline \hline
    Total & 23320  \\ \hline     
    \end{tabular}%
  \label{tab:ioCurrents}%
\end{table}%
Assuming an IO current of 23.32mA and an average supply current of 1.6mA, the total power dissipation can then be calculated to be:
\[P_{diss} = V_{DD} \cdot \left(1.6m + 23.32m \right)\]
\[\boxed{P_{diss} = 82.24mW}\]

Using $\theta_{JA} = 47.8^\circ C/W$, the total junction temperature is then given by:

\[T_J = 27 + 26.8 \cdot .08224 \]
\[\boxed{T_J = 29.20 ^\circ C}\]

The datasheet states that the maximum junction temperature is $95^\circ C$, which means that in this application, the device is well under the maximum operating temperature rating and no additional heat dissipation mechanism is needed.


\subsection{Low-Dropout Regulator}
The datasheet for the TPS73501 contains a section on power dissipation and provides the following formula for calculating the power dissipation across the device:

\[P_{diss} = \left(V_{IN} - V_{OUT}\right)\cdot I_{OUT}\]

The voltage provided by the battery charger circuit is 3.925 V and the required output voltage is 3.3 V.  The estimated output current required is 180.96 mA.  In order to leave a margin of error, a rounded value of 200 mA will be used in this calculation.  Thus, the power dissipated by the device is given by:

\[P_{diss} = \left(3.925 - 3.3\right)\cdot 200m\]
\[\boxed{P_{diss} = 125mW = 0.125W}\]

Since for this device, $\theta_{JA} = 47.8^\circ C/W$, the total junction temperature is then given by:

\[T_J = 27 + 47.8 \cdot 0.125 \]
\[\boxed{T_J = 32.98 ^\circ C}\]

The datasheet states that the maximum junction temperature is $150^\circ C$, which means that in this application, the device is well under the maximum operating temperature rating and no additional heat dissipation mechanism is needed.


\subsection{Battery Charger}

The datasheet for the BQ24072 provides the following formula for estimating the power dissipation across the device.
\[P_{diss} = \left(V_{IN} - V_{OUT}\right)\cdot I_{OUT} + \left(V_{OUT} - V_{BAT}\right)\cdot I_{BAT}\]

The output voltage of the device is 5.5 V, while the input voltage is 5 V as dictated by the USB standard.  The output current can be estimated to be 50mA by taking into account the two LEDs connected to this pin (at about 20mA per LED) and the LDO input (46 $\mu$A) and rounding up to leave a margin for safety.  The circuit was designed for a battery voltage of 3.7 V and a current of 800 mA.

\[P_{diss} = \left(5 - 3.925\right)\cdot 50m + \left(3.925 - 3.7\right)\cdot 500m\]
\[\boxed{P_{diss} = 166 mW}\]

Since for this device, $\theta_{JA} = 39.47^\circ C/W$, the total junction temperature is then given by:

\[T_J = 27 + 39.47 \cdot 0.166 \]
\[\boxed{T_J = 33.55 ^\circ C}\]

The datasheet states that the maximum junction temperature is $125^\circ C$, which means that in this application, the device is well under the maximum operating temperature rating and no additional heat dissipation mechanism is needed.


\subsection{Battery Meter}
Although the datasheet for the BQ27410 does not provide a direct formula for power dissipation and because the battery meter has an internal LDO, the formula used for the MCU IO pins will be combined with the power dissipation formula of the LDO to obtain a more robust estimate for the power dissipated.  

\[P_{diss} = V_{DD} \cdot \left(\sum_{allpins} |I_{IO(avg)}| \right) + \left(V_{IN} - V_{OUT}\right)\cdot I_{OUT}\]

The IO pin output current for this device is between 0.5 and 1 mA.  1 mA will be used for the two $I^2C$ pins, which gives a total IO current of 2 mA.  The maximum supply current is 103 $\mu$A and this will be used instead of the average supply current to leave a margin of safety.  The input voltage is 5.5 V and the regulated output voltage is 2.5 V.

\[P_{diss} = 2.5 \cdot \left(2m \right) + \left(3.7 - 2.5\right)\cdot 0.103m\]

\[\boxed{P_{diss} = 5.12 mW}\]

Since for this device, $\theta_{JA} = 64.17^\circ C/W$, the total junction temperature is then given by:

\[T_J = 27 + 64.1 \cdot 0.00512 \]
\[\boxed{T_J = 27.33 ^\circ C}\]

The datasheet states that the maximum junction temperature is $100^\circ C$, which means that in this application, the device is well under the maximum operating temperature rating and no additional heat dissipation mechanism is needed.

