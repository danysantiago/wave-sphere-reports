\section{Introduction}
%The background
The waves are abundant at sea, yet it is known little about them. The waves break at shore and at sea, moving with dynamics that have not been understood precisely. This project intends to help researchers on their journey of discovering the natural physics and motion dynamics of waves. As the waves break, the water particles create twists and turns above and below them, sometimes forming vortices. These particle movements are the ones that describe the dynamics of the waves. The best way to study such dynamics is by imitating these particles \cite{Amador2012}. Hence, this work proposes a device that somewhat mimics these particles.

Upon consultation with Dr.~Miguel Canals, an assistant professor of Fluid Mechanics and Coastal Engineering at the University of Puerto Rico at Mayag\"uez and Director of the Fluid Mechanics and Ocean Engineering Laboratories, he presented the authors of this proposal with a problem currently being faced by his research team. Currently, his team is working on an NSF funded project titled ``Lagrangian Observations of Turbulence in Breaking Waves", for which they have designed capsules to collect the data needed for their experiments \cite{Canals2012}.  The capsules that they have designed, although functional, could be greatly improved. The capsules\footnote{From hereon the terms ``capsules'' and ``spheres'' will be used interchangeably.} have limited capabilities that made the data collecting process difficult. A physical button outside the spherical casing was used to turn the capsule on which created an easy access for the water to enter inside the capsule, damaging the electrical components. A single sensor was present in the sphere, a triaxial accelerometer, which limited the data they could gather during an experiment. Moreover, because the size of the capsule is quite small, the spheres could be easily lost and needed to be tied so that they could be easily found, an action which might interfere with the integrity of the data being collected. The research team uses the capsules to perform the experiments that collect the necessary data. An experiment is performed by throwing the capsules into identified waves and gathering their data trough the capsule's sensor for at least thirty seconds. Once the thirty seconds had passed the wave would have finished and the sphere would eventually drift to the shore.

%The purpose
This work aims to aid the same group of researchers at the University of Puerto Rico at Mayag\"uez to create an improved implementation of the Lagrangian\footnote{The Lagrangian, L, of a dynamical system is a function that summarizes the dynamics of the system.} drifters. This work proposes to redesign and improve their current implementation by adding several features and components.  This includes adding more sensors to obtain other data such as a gyroscope, magnetometer and GPS. Implementing a base station that attaches to a computer and can communicate wirelessly with the capsules to solve the data retrieval problem and possibly the localizing problem. All while at the same time handling components to conserve energy for a successful day of experiments.

%The scope
In order for the improved implementation to be successful it must still adhere to the constraints established in the original idea, that is, the capsule must have a spherical form and must be as small as possible, trying to imitate a water particle. Moreover, because more than one capsule will be used for an experiment they must also be cost effective.  This work aims to create capsules that are in the same price range or lower than the previously created one by the research team while adding the extra features previously mentioned. The goal is an overall improvement of their solution so that they can use this system to obtain better results from their experiments.