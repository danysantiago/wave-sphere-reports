\section{Design Criteria}
\begin{enumerate}
\item Power and Energy
	\begin{itemize}
	\item The sphere must be as energy efficient as possible, since it must endure long days on the field with a very limited space for storing energy. The limited space for energy storage can be remedied by the use of energy harvesting techniques. Since the sphere will be rocked by the waves, the energy harvesting techniques are viable.
	\end{itemize}
\item Sustainability
	\begin{itemize}
	\item To be cost effective, the sphere should endure as long as possible. The chassis design must be strong and rugged to resist heavy shock from the waves and possible rocks. The internal electrical components should be of the most recent models so that they reach their obsolescence phase in a longer time. The internal components should also be tightened securely to the chassis so that they do not get damaged by bumping against the sphere chassis. Since multiple uses is expected from the sphere, extra care will be taken to design a long lasting and long cycle power system.
	\end{itemize}
\item System Integrity
	\begin{itemize}
	\item Integrity of the data is of utmost important in the application in which the spheres are intended to be used. The chassis design is an important criteria for ensuring data integrity, since the internal memory and measurement components could be damaged by a faulty chassis that did not absorb shock correctly. 
	\end{itemize}
\end{enumerate}