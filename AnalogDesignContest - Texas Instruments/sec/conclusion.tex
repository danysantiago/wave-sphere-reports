\section{Conclusion}

The proposed 7.5cm drifter was successfully designed and implemented.  The space constraint, which proved to be the most prohibitive and difficult to adhere to constraint was satisfied by introducing a two-tiered printed circuit board (PCB) structure consisting of two individual PCBs connected through a board-to-board connector.  The newly introduced components solve certain problems that had been previously encountered by the researchers.  For example, the on-board GPS solves the problem of the drifters getting lost at sea with no way to locate them, while the RF wakeup module solves the water leakage problem introduced when attempting to mount an external push button to activate the sphere.  From a data collecting perspective, a faster and steady sampling rate was achieved (250Hz) in comparison to the existing prototype being used by the researchers.  

An application in the form of a Graphical User Interface (GUI) was successfully implemented and tested.  This allows the researchers to manage the drifters as mentioned throughout this work.  Capabilities for managing multiple drifters at the same time were implemented and are available for the users.