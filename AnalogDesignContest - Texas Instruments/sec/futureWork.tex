\section{Future Work}
\label{sec:futureWork}
The calibration of the sensors can be different from board to board because of the different amounts of stress caused by the soldering process.  Since data conversion takes place in the Graphical User Interface this means that a calibration mode should be added to the software.  The software should take at least two parameters for each sensor axis: a scaling factor and an offset.  Although there are very elaborated calibration processes and methods, these two parameters are essential to any calibration process. 

From a software perspective, more measures to protect data integrity should be implemented since for this type of application having corrupted data is unacceptable.  Some measures might include storing data redundantly, oversampling and averaging values on other sensors, among others.

From a hardware perspective, a shock-resistant gyroscope should replace the one used in the current design.  The battery meter is yet to be fully connected and the software needed to communicate with this component is yet to be fully implemented.  This component has been included throughout the design process but since it is not a system critical component, it has not been completely integrated at the time this work was published.  In addition to this, since the first level PCB has not arrived from the fabrication company at the time this work was published, all components still need to be soldered on to the board and tested.  The second level PCB arrived shortly before the deadline imposed on this work and therefore all components still need to be soldered as well.