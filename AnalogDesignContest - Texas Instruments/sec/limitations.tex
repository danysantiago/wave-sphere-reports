\section{System Limitations}

\subsection{Hardware Limitations}
During the late stages of the prototype development one of the researchers discovered from his gathered data that the gyroscope was very sensible to vibrations and changes in acceleration, something that will always occur because of the nature of the experiments being conducted with these drifters.  The gyroscope used for this prototype suffers from the same negative effect as the one the researcher is using.  Because of the advanced stage of development, it became unfeasible to change the selected gyroscope to one that was more resistant to these effects. Therefore, in a future version of the prototype a new gyroscope that is shock resistant needs to be chosen in order to yield better results.

\subsection{Software Limitations}
Due to the time constraints imposed on the project, at the time this work was published, one of the biggest limitations in terms of software is the inability to calibrate the sensors without having to re-compile the code.  The effect of sensor calibration is seen when the raw data obtained from the sensors is converted into real data with physical meaning.  For each measurement, a minimum of scaling and offset factors should be included and are currently not present.  This has also been recommended as a possible extension of the software in Section~\ref{sec:futureWork}. However, since sensor calibration can be different between board to board due to the stress and strain caused when the surface-mounted components are soldered, it can be argued that it is best for the system to save and manage raw data.  This data can later be converted into physical values by a separate data analysis software that can take the calibration parameters as input.